\documentclass[./standalone.tex]{subfiles}
%\documentclass[../../../CR/pac.tex]{subfiles}

\begin{document}




%%%% ~~~~~~~~~~~~~~~~~~~~~~~~~~~~~~~~~~~~~~~~~~~~~~~~~~~~~~~~~~~~%%%%
%%%% //////////////////////// PARTIE 1 \\\\\\\\\\\\\\\\\\\\\\\\\\%%%% 
%%%% ~~~~~~~~~~~~~~~~~~~~~~~~~~~~~~~~~~~~~~~~~~~~~~~~~~~~~~~~~~~~%%%%
\part{Analyse de billet de concert}

%%% ======================================= %%%
%%%                EXERCICE 1               %%%
%%% ======================================= %%%
\section{Exercice 1: Traitement d'une commande de billets}


%% ---- 1)
\subsubsection{Spécification de Concert.1.lex: reconnaissance des champs clefs}
\lstinputlisting[style=C, caption=Première spécification en vue d'un test de reconnaissance des différents champs d'une commande de billets]{../../../codes/Ex1/Concert.1.lex}
\newpage

%% ---- 2)
\subsubsection{Spécification de Concert.2.lex: application de la reconnaissance à un besoin 'réel'}
\lstinputlisting[style=C, caption=Seconde spécification appliquant la reconnaissance des différents champs d'une commande de billets]{../../../codes/Ex1/Concert.2.lex}





%%%% ~~~~~~~~~~~~~~~~~~~~~~~~~~~~~~~~~~~~~~~~~~~~~~~~~~~~~~~~~~~~%%%%
%%%% //////////////////////// PARTIE 2 \\\\\\\\\\\\\\\\\\\\\\\\\\%%%% 
%%%% ~~~~~~~~~~~~~~~~~~~~~~~~~~~~~~~~~~~~~~~~~~~~~~~~~~~~~~~~~~~~%%%%
\part{Des automates en récursif}

%%% ======================================= %%%
%%%                EXERCICE 2               %%%
%%% ======================================= %%%
\section{Exercice 2: Programmation en dur de manière récursive}
\bigskip

%% ---- 1)
\subsubsection{Questions de compréhension}
\medskip

\textbf{Question:} Si votre automate a N états, combien de fonctions reconnaitRec\_i devez vous écrire?\\

\textbf{Réponse:} Si l'automate a N états alors il faudra écrire N fonctions reconnaitRec\_i. En effet, dans les faits nous sommes en train d'implanter un système d'équations aux langages.\\\\


\textbf{Question:} Si l'état i est final, que doit retourner reconnaitRec\_i("")? Et si i n'est pas final?\\

\textbf{Réponse:} Un état i final signifie que reconnaitRec\_i("") doit retourner 'true', "" étant le mot vide aussi appelé $\epsilon$. Tout état i non final doit alors retourner 'false' pour le mot vide.\\\\


\textbf{Question:} Si le paramètre 'mot' n'est pas vide et commence par un caractère c, quelle fonction reconnaitRec\_i(mot) doit-elle appeler? Et avec quel paramètre?\\

\textbf{Réponse:} Si le paramètre 'mot' n'est pas vide et commence par un caractère c alors on doit appeler la fonction reconnaitRec\_i(mot) qui correspond à l'état de destination dans la transition $q_{courant} \xrightarrow{c}  q_i$. On appelle alors cette fonction avec pour paramètre le mot 'mot' tronqué de sa première lettre.\\\\


%% ---- 2)
\subsubsection{Automate des réels}
\medskip
\begin{center}
	\includegraphics[scale=0.5]{../VP/ex2_2.jpg}
\end{center}
\newpage

%% ---- 3)
\subsubsection{Implantation des reconnaitRec}
\lstinputlisting[style=Ocaml, caption=Début du code source d'automateEnDurReels.ml, firstline=10, lastline=51]{../../../codes/Ex2/automateEnDurReels.ml}
\newpage

%% ---- 4)
\subsubsection{Implantation de l'automate complet}
\lstinputlisting[style=Ocaml, caption=Le reste du code source d'automateEnDurReels.ml, firstline=53]{../../../codes/Ex2/automateEnDurReels.ml}


%% ---- 5)
\subsubsection{Programme complet}
\lstinputlisting[style=Ocaml, caption=programme final lisant sans cesse sur le flux d'entrée]{../../../codes/Ex2/programmeFinal.ml}
\newpage




%%%% ======================================= %%%
%%%%                EXERCICE 3               %%%
%%%% ======================================= %%%
\section{Exercice 3: Des automates non déterministes représentés dans le code de manière récursive}
\bigskip

\begin{center}
	\includegraphics[scale=0.5]{../VP/ex3.jpg}
\end{center}

%%% ---- 1)
\subsubsection{Questions de compréhension}
\medskip

\textbf{Question:} Comment dans le code de reconnait\_0 allez vous représenter le fait qu'en lisant un a, on puisse aller soit de l'état 0 à l'état 1, soit de l'état 0 à l'état 2 ?\\

\textbf{Réponse:} Un automate non déterministe signifie que l'on peut trouver un chemin valide (et donc reconnaître un mot) en passant soit par un chemin soit par un autre. Ce soit/soit ce représente en programmation par l'opérateur OU. Je vais donc faire pour la lettre 'a' à l'état 0: reconnaitRec1 resteDuMot || reconnaitRec2 resteDuMot\\\\


\textbf{Question:} Comment dans le code de reconnait\_1 allez vous représenter le fait que l'on peut passer directement, sans rien lire, à l'état 2 ?\\

\textbf{Réponse:} Similairement au cas si dessus je peux faire un OU avec directement l'appel récursif de reconnaitRec2 en donnant en paramètre de reconnaitRec2 le mot complet sans avoir "mangé" de lettre.\\\\
\newpage


%%% ---- 2)
\subsubsection{Implantation et Tests de l'automate}
\lstinputlisting[style=Ocaml, caption=Code source d'automateEnDurEx3.ml, firstline=10, lastline=52]{../../../codes/Ex3/automateEnDurEx3.ml}
\newpage




%%%% ======================================= %%%
%%%%                EXERCICE 4               %%%
%%%% ======================================= %%%
\section{Exercice 4: Évaluation du réel correspondant à la chaîne de caractères}
\bigskip

%%% ---- 1)
\subsubsection{Questions de compréhension}
\medskip

\textbf{Question:} Comment allez vous gérer votre position dans la partie décimale ($x^{eme}$ position après la virgule = indique la puissance de dix négative?), et vous en servir pour prendre en compte la nouvelle décimale lue? \\

\textbf{Réponse:} \\\\


\textbf{Question:} Comment allez vous gérer le calcul de la partie entière, lorsqu'un nouveau chiffre est lu? \\

\textbf{Réponse:} \\\\

\textbf{Question:} Comment allez vous gérer la transmission du calcul d'une routine récursive à l'autre? Variables globales, paramètres d'entrée sortie, valeurs de retour de fonction?\\

\textbf{Réponse:} \\\\

%%% ---- 2)
\subsubsection{Implantation de la fonction d'évaluation des réels}

%%% ---- 3)
\subsubsection{Implantation du programme final d'évaluation des réels}


\end{document}